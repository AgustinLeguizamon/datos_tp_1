\documentclass[12pt,a4paper,draft]{article}
\title{Organizacion de Datos - TP1}

\begin{document}

\maketitle
\newpage
\tableofcontents

\newpage
\section{Introducción}

\subsection{Objetivos}
El presente informe contiene un análisis exploratorio como primer trabajo practico grupal de la materia, hecho sobre un set de datos obtenido a partir de Twitter, con sus respectivas visualizaciones y conclusiones. \\[0.25cm]

En dicho set, las entradas se obtienen a partir de una lista de tweets que podrían o no hablar sobre una catástrofe, junto con otros atributos que los caracterizan. Buscaremos, luego de un análisis de los datos, si existe alguna relación directa entre las variables que nos presentan o si se puede obtener alguna conclusión interesante sobre los datos que podemos extraer de dichas variables.

\subsection{Set de datos}
El set de datos dado para el análisis consta de registros que representan cada uno de los tweets, con la siguiente estructura:
\begin{itemize}
\item id - identificador unico para cada  tweet
\item text - el texto del tweet
\item location - ubicación desde donde fue enviado (podría no estar)
\item keyword - un keyword para el tweet  (podría faltar)
\item target - en train.csv, indica si se trata de un desastre real  (1) o no (0)
\end{itemize}


\newpage
\section{Análisis del set de datos}

\subsection{Análisis general}

\subsection{Análisis sobre el texto del tweet}

\subsubsection{Largo de los tweets}

\subsubsection{Largo de palabras}

\subsubsection{Palabras en común}

\subsubsection{Hashtags}

\subsubsection{Etiquetas}

\subsubsection{Links}

\subsubsection{Fechas y horarios}

\subsection{Análisis sobre palabras clave}

\subsection{Análisis sobre la localización de los tweets}

\subsection{Otros Análisis}

\section{Conclusiones}






\end{document}